\documentclass{article}
\usepackage{graphicx} % Required for inserting images
\usepackage{amsmath} % align equals

\title{Project 7 Cryptography}
\author{Caden Kim, Brennan Longstreth, Riley Sikes }
\date{March 24 2023}

\begin{document}

\maketitle

\section{Homework 7}

%%%%%%%%%%% PROBLEM 1 %%%%%%%%%%%
%%%%%%% Project 6, Problem 4 asked that you generate a public key and submit it to me in a file called public_key.txt.  Very soon, possibly even today (4/7), I’ll send you via email an encrypted message.  The answer to this problem is its decryption.  The email will go to the first author for the github repository that you used for problem 6.
\subsection{Decryption}

%%%%%%%%%%% PROBLEM 2 %%%%%%%%%%%
%%%%% Hint for problems 2 - 4:
% The obvious approach is to do a brute force search of all relevant exponents, stopping at the appropriate time.  You may write a Sage list comprehension  to produce a list of  tuples of the form: (n,q^n mod k) where q is the integer whose order you are looking for, n is the exponent, and k is the modulus.  You can determine by inspection the order of n.   If you use the brute force technique, show all computations.  If you use the sage technique, show the single line list comprehension.  In either case, make an argument.
\subsection{Order of 2 mod 17}

%%%%%%%%%%% PROBLEM 3 %%%%%%%%%%%
\subsection{Order of 3 mod 19}

%%%%%%%%%%% PROBLEM 4 %%%%%%%%%%%
\subsection{Order of 5 mod 23}

%%%%%%%%%%% PROBLEM 5 %%%%%%%%%%%
\subsection{Proof}
If $a$ has order $hk$ mod $n$ then $a^{h}$ has order $k$ mod $n$
\\\\
Suppose $a$ has order $hk$ mod $n$, then by the definition of order we can say
$$a^{hk} \equiv 1 \pmod{n}$$
Observe that $a^{hk} \equiv a^{h^{k}} \equiv 1 \pmod{n}$
\\\\
Therefore $a^{h}$ has order $k$ mod $n$

%%%%%%%%%%% PROBLEM 6 %%%%%%%%%%%
%%%%%%%%%% You’ll find the short-cut theorem useful.
\subsection{Proof}
The odd prime divisors of the integer $n^4 +1$ are of the form $8k + 1$.  

%%%%%%%%%%% PROBLEM 7 %%%%%%%%%%%
%%%%%%%%%%%%% You’ll also find the  algorithm in McAndrew, p. 119.  Show all work.
\subsection{Proof}
Using the primitive root test algorithm developed in class, find the primitive roots of 13.



\end{document}
