\documentclass{article}
\usepackage{graphicx} % Required for inserting images
\usepackage{amsmath} % align equals

\title{Project 7 Cryptography}
\author{Caden Kim, Brennan Longstreth, Riley Sikes }
\date{March 24 2023}

\begin{document}

\maketitle

\section{Homework 7}

%%%%%%%%%%% PROBLEM 1 %%%%%%%%%%%
%%%%%%% Project 6, Problem 4 asked that you generate a public key and submit it to me in a file called public_key.txt.  Very soon, possibly even today (4/7), I’ll send you via email an encrypted message.  The answer to this problem is its decryption.  The email will go to the first author for the github repository that you used for problem 6.
\subsection{Decryption}
From your first email: crueleest
From your corrected email: (latex crashes if I paste it here)
%%%%%%%%%%% PROBLEM 2 %%%%%%%%%%%
%%%%% Hint for problems 2 - 4:
% The obvious approach is to do a brute force search of all relevant exponents, stopping at the appropriate time.  You may write a Sage list comprehension  to produce a list of  tuples of the form: (n,q^n mod k) where q is the integer whose order you are looking for, n is the exponent, and k is the modulus.  You can determine by inspection the order of n.   If you use the brute force technique, show all computations.  If you use the sage technique, show the single line list comprehension.  In either case, make an argument.
\subsection{Order of 2 mod 17}
$$2^1 \equiv 2 mod 17$$
$$2^2 \equiv 4 mod 17$$
$$2^3 \equiv 8 mod 17$$
$$2^4 \equiv 16 mod 17$$
$$2^5 \equiv 15 mod 17$$
$$2^6 \equiv 13 mod 17$$
$$2^7 \equiv 9 mod 17$$
$$2^8 \equiv 1 mod 17$$
The order of 2 mod 17 is 8

%%%%%%%%%%% PROBLEM 3 %%%%%%%%%%%
\subsection{Order of 3 mod 19}
$$3^1 \equiv 3 mod 19$$
$$3^2 \equiv 9 mod 19$$
$$3^3 \equiv 8 mod 19$$
$$3^4 \equiv 5 mod 19$$
$$3^5 \equiv 15 mod 19$$
$$3^6 \equiv 7 mod 19$$
$$3^7 \equiv 2 mod 19$$
$$3^8 \equiv 6 mod 19$$
$$3^9 \equiv 18 mod 19$$
$$3^10 \equiv 16 mod 19$$
$$3^11 \equiv 10 mod 19$$
$$3^12 \equiv 11 mod 19$$
$$3^13 \equiv 14 mod 19$$
$$3^14 \equiv 4 mod 19$$
$$3^15 \equiv 12 mod 19$$
$$3^16 \equiv 17 mod 19$$
$$3^17 \equiv 13 mod 19$$
$$3^18 \equiv 1 mod 19$$
The order of 3 mod 19 is 18

%%%%%%%%%%% PROBLEM 4 %%%%%%%%%%%
\subsection{Order of 5 mod 23}
$$5^1 \equiv 5 mod 23$$
$$5^2 \equiv 2 mod 23$$
$$5^3 \equiv 10 mod 23$$
$$5^4 \equiv 4 mod 23$$
$$5^5 \equiv 20 mod 23$$
$$5^6 \equiv 8 mod 23$$
$$5^7 \equiv 17 mod 23$$
$$5^8 \equiv 16 mod 23$$
$$5^9 \equiv 11 mod 23$$
$$5^10 \equiv 9 mod 23$$
$$5^11 \equiv 22 mod 23$$
$$5^12 \equiv 18 mod 23$$
$$5^13 \equiv 21 mod 23$$
$$5^14 \equiv 13 mod 23$$
$$5^15 \equiv 19 mod 23$$
$$5^16 \equiv 3 mod 23$$
$$5^17 \equiv 15 mod 23$$
$$5^18 \equiv 6 mod 23$$
$$5^19 \equiv 14 mod 23$$
$$5^20 \equiv 3 mod 23$$
$$5^21 \equiv 13 mod 23$$
$$5^22 \equiv 1 mod 23$$
The order of 5 mod 23 is 22

%%%%%%%%%%% PROBLEM 5 %%%%%%%%%%%
\subsection{Proof}
If $a$ has order $hk$ mod $n$ then $a^{h}$ has order $k$ mod $n$
\\\\
Suppose $a$ has order $hk$ mod $n$, then by the definition of order we can say
$$a^{hk} \equiv 1 \pmod{n}$$
Observe that $a^{hk} \equiv a^{h^{k}} \equiv 1 \pmod{n}$
\\\\
Therefore $a^{h}$ has order $k$ mod $n$

%%%%%%%%%%% PROBLEM 6 %%%%%%%%%%%
%%%%%%%%%% You’ll find the short-cut theorem useful.
\subsection{Proof}
The odd prime divisors of the integer $n^4 +1$ are of the form $8k + 1$.  
Suppose that $p$ is an odd divisor of $n^4 + 1$, so that $n^4 \equiv -1$ mod $p$. This implies that $n^8 \equiv 1$ mod $p$. Euler's theorem tells us that $8^{\phi (p)} \equiv 1$ mod $p$, in other words, $8^{p-1} \equiv 1$ mod $p$. Hence it follows that $8 | (p - 1)$. In other words, $p = 8k + 1$ for some $k$. Therefore, the odd prime divisors of the integer $n^4 + 1$ are of the form $8k + 1$.

%%%%%%%%%%% PROBLEM 7 %%%%%%%%%%%
%%%%%%%%%%%%% You’ll also find the  algorithm in McAndrew, p. 119.  Show all work.
\subsection{Proof}
Using the primitive root test algorithm developed in class, find the primitive roots of 13.
\\\\
First lets see how many primitive roots 13 has. We can find this out by doing $\phi(\phi(13))$ which gives us 4
\\\\
To find the primitive roots, we only need to test values less than 13 which are coprime with 13 which are A = (1, 2, 3, 4, 5, 6, 7, 8, 9, 10, 11, 12)
\\\\
For any value a in A to be a primitive root of 13, $a^{\phi(13)/d} \not \equiv 1 \pmod{13}$ for all values d. d is the factors of 12 which are $2^2 * 3$, so d = (2, 3). Thus the powers we need to test are $12/2 = 6$ and $12/3 = 4$.
\\\\
Hence for each value a in A, we need to test that $a^{4} \not \equiv 1 \pmod{13}$ and $a^{6} \not \equiv 1 \pmod{13}$
\begin{enumerate}
    \item $1^4 \equiv 1 \pmod{13}$ so NOT a PR
    \item $2^4 \equiv 3 \pmod{13}$ and $2^6 \equiv 12 \pmod{13}$ so IS a PR
    \item $3^4 \equiv 3 \pmod{13}$ and $3^6 \equiv 1 \pmod{13}$ so NOT a PR
    \item $4^4 \equiv 9 \pmod{13}$ and $4^6 \equiv 1 \pmod{13}$ so NOT a PR
    \item $5^4 \equiv 1 \pmod{13}$ so NOT a PR
    \item $6^4 \equiv 9 \pmod{13}$ and $6^6 \equiv 12 \pmod{13}$ so IS a PR
    \item $7^4 \equiv 9 \pmod{13}$ and $7^6 \equiv 12 \pmod{13}$ so IS a PR
    \item $8^4 \equiv 1 \pmod{13}$ so NOT a PR
    \item $9^4 \equiv 9 \pmod{13}$ and $9^6 \equiv 1 \pmod{13}$ so NOT a PR
    \item $10^4 \equiv 3 \pmod{13}$ and $10^6 \equiv 1 \pmod{13}$ so NOT a PR
    \item $11^4 \equiv 3 \pmod{13}$ and $11^6 \equiv 12 \pmod{13}$ so IS a PR
    \item $12^4 \equiv 1 \pmod{13}$ so NOT a PR
\end{enumerate}
Thus the four primitive roots of 13 are 2, 6, 7, and 11.


\end{document}